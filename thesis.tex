%% -----------------------------------------------------------------
%% This file uses UTF-8 encoding
%%
%% For compilation use following command:
%% latexmk -pdf -pvc -bibtex thesis
%%
%% -----------------------------------------------------------------
%%                                     _     _      
%%      _ __  _ __ ___  __ _ _ __ ___ | |__ | | ___ 
%%     | '_ \| '__/ _ \/ _` | '_ ` _ \| '_ \| |/ _ \
%%     | |_) | | |  __/ (_| | | | | | | |_) | |  __/
%%     | .__/|_|  \___|\__,_|_| |_| |_|_.__/|_|\___|
%%     |_|                                          
%%
%% -----------------------------------------------------------------

\documentclass{kithesis}

% Additional packages
\usepackage[main=slovak,english]{babel}
% For thesis written in English just change the order of languages:
% \usepackage[main=english,slovak]{babel}

\usepackage{todonotes}
\usepackage{threeparttable}

\usepackage{listings}  % for source code
% Listings settings
% See for details: https://en.wikibooks.org/wiki/LaTeX/Source_Code_Listings
\lstset{
    basicstyle=\small\ttfamily,  % smaller typewriter font
    showstringspaces=false       % don't show spaces in string
}

% Location of file with bibliography resources
\addbibresource{chapters/bibliography.bib}

\newcommand{\detailtexcount}[1]{%
  \immediate\write18{texcount -merge -sum -q #1.tex output.bbl > #1.wcdetail }%
  \verbatiminput{#1.wcdetail}%
}

\newcommand{\quickwordcount}[1]{%
  \immediate\write18{texcount -1 -sum -merge -q #1.tex output.bbl > #1-words.sum }%
  \input{#1-words.sum} words%
}

\newcommand{\quickcharcount}[1]{%
  \immediate\write18{texcount -1 -sum -merge -char -q #1.tex output.bbl > #1-chars.sum }%
  \input{#1-chars.sum} characters (not including spaces)%
}

% Variables
%\thesisspec{figures/thesisspec.png} 

\title{My thesis \br (the skeleton)}{Web crawler pre oblasť internetových správ}

\author{Peter}{Makláry}
\supervisor{Michal Solanik} %veduci prace
%\consultant{Donald E. Knuth} %konzultant
%\college{University of Žilina}{Žilinská univerzita} %univerzita
%\faculty{Faculty of Electrical Engineering and informatics}{Fakulta elektrotechniky a informatiky} %fakulta
%\department{Department of Computers and Informatics}{Katedra počítačov a informatiky} %katedra
%\departmentacr{DCI}{KPI} % skratka katedry
%\thesis{Master thesis}{Diplomová práca} %typ prace
\submissiondate{26}{5}{2023}
%\fieldofstudy{9.2.1 Informatika}
%\studyprogramme{Informatika}
%\city{Košice} %mesto
\keywords{web crawler, data collection, news webs}{web crawler, zber dát, spravodajské weby}
%\declaration{som nepodvadzal}

\abstract{%
    % english 
	This thesis presents the design and implementation of a specialized web crawler for Slovak news websites. Its key advantages include low maintenance and operational requirements, as well as full control over the crawling process. The crawler demonstrates resilience and can operate for extended periods without losing progress in case of failure. Using a modular architecture, the crawler allows easy customization of its behavior. Runtime statistics are collected to monitor the crawling process effectively. In testing, the crawler collected 9 gigabytes of data from 518,000 web pages within 11 hours. 
}{%
    % slovak 
	Táto práca predstavuje návrh a implementáciu špecializovaného webového crawlera pre zber dát zo slovenských spravodajských webov. Jeho hlavnými výhodami je nízka náročnosť na údržbu a prevádzku a plná kontrola nad procesom crawlovania. Crawler preukazuje odolnosť a dokáže dlhodobo fungovať bez straty stavu crawlovania v prípade pádu. Pomocou modulárnej architektúry je možné jednoducho prispôsobiť správanie crawlera. Zbiera štatistiky behu na efektívne monitorovanie crawlovacieho procesu. Počas testovania crawler zozbieral 9 gigabajtov dát z 518 000 webových stránok za 11 hodín. 
}

\thesisspec{figures/zadavaci-list.png}

\acknowledgment{Na tomto mieste by som rád poďakoval môjmu vedúcemu práce, Ing. Michalovi Solanikovi, za jeho čas, cenné rady a odborné vedenie počas riešenia mojej záverečnej práce.

Rovnako by som sa rád poďakoval svojim rodičom a priateľom za ich nepretržitú podporu a povzbudzovanie počas celého môjho štúdia. Chcel by som osobitne poďakovať môjmu otcovi za naše dlhé rozhovory a pomoc nielen pri štúdiu, ale aj pri vypracovaní tejto práce. Vypestoval vo mne zvedavosť a radosť z riešenia rébusov a problémov, čím mi otvoril dvere do sveta nekončiaceho učenia. Za to mu patrí moja veľká vďaka. 
}
% if you want to work only on selected chapters
% \includeonly{chapters/analysis} %,chapters/synteza}
% \includeonly{chapters/synthesis} %,chapters/synteza}

% Load acronyms
\input{acronyms}


%% -----------------------------------------------------------------
%%          _                                       _   
%%       __| | ___   ___ _   _ _ __ ___   ___ _ __ | |_ 
%%      / _` |/ _ \ / __| | | | '_ ` _ \ / _ \ '_ \| __|
%%     | (_| | (_) | (__| |_| | | | | | |  __/ | | | |_ 
%%      \__,_|\___/ \___|\__,_|_| |_| |_|\___|_| |_|\__|
%%                                                      
%% -----------------------------------------------------------------

\begin{document}
%% Title page, abstract, declaration etc.:
\frontmatter{}

\lstset{language=Scala}

%% List of code listings, if you are using package minted
%\listoflistings

%\pagenumbering{arabic}

%% Chapters
% !TEX root = ../thesis.tex

\chaptermark{Úvod}
\phantomsection
\addcontentsline{toc}{chapter}{Úvod}

\chapter*{Úvod}

S nástupom digitálnej doby sa objem dostupných informácii exponenciálne zväčšuje. Vzniklo množstvo spravodajských webov, publikujúcich násobne väčšie množstvo článkov než ich predchodcovia - printové média. Dokážu sprostredkovať udalosti z celého sveta na minútovej báze. S narastajúcim objemom dát je za účelom získania vedomostí potrebné efektívne zozbieranie a spracovanie dát. Niekedy to bola úloha knižníc, dnes na to čisto ľudská sila nestačí. Držať krok s digitálnou dobou dokážeme, iba ak ju využijeme v náš prospech. Práve program nazývaný web crawler zohráva kľúčovú úlohu pri zbieraní dát zo spravodajských webov. 

Web crawler umožňuje automatizované zbieranie a organizovanie dát zo širokého spektra zdrojov. Minimalizuje potrebu manuálnej práce, čím ušetrí čas a náklady. Spravodajské weby väčšinou neposkytujú zoznam všetkých ich článkov, tie sú rozptýlené a navzájom poprepájané odkazmi. Crawler umožňuje automatizovane prejsť túto štruktúru a získať o nej informácie. Takto teda vieme zaznamenať aj vzťahy medzi článkami, nie len ich samotný obsah. 

Zbieranie takýchto dát otvára možnosti hĺbkovým analýzam. Z extrahovaných informácii vieme analýzou identifikovať vzorce, trendy a korelácie v spravodajstve. To nám poskytne hodnotný náhľad do mediálneho priestoru. Napríklad hodnotenie vplyvu konkrétnych udalostí, odhaľovanie predsudkov, sledovanie šírenia dezinformácií alebo pochopenie nálad verejnosti voči určitým témam. 

Existuje množstvo dostupných crawlerov. Poväčšine sú ale na účely získania vedomostí z priestoru slovenského spravodajstva príliš robustné a komplexné. Majú vlastnosti, ktoré nepotrebujeme a vyžadujú si priveľké zdroje na prevádzku. Alebo sú síce jednoduché, ale zato málo prispôsobiteľné pre naše potreby. 

Cieľom práce je navrhnutie a implementovanie čo najmenej komplexného crawlera, zbierajúceho dáta zo slovenských spravodajských webov. Musí byť nenáročný na prevádzku a pritom byť nastaviteľný podľa potrieb analýzy, ako sú zmena schémy výstupných dát, pridávanie a odoberanie zdrojov. Vlastná implementácia nám dáva možnosť prispôsobiť ho presne pre naše potreby.


% Základom každej analýzy sú kvalitné dáta. Jedným zo zdrojov takýchto dát sú web stránky, kde sú však rozptýlene a uložené v nevhodnej podobe. Tento problém rieši webový crawler.

% Motiváciou práce je príprava dát zo slovenských spravodajských webov na následnú analýzu. Existuje viacero crawlerov, nastaviteľných na získavanie dát pre tieto cieľové účely. Sú ale príliš robustné a vyžadujú si zbytočne veľa zdrojov na prevádzku. 

% Prínosom tejto práce je navrhnutie vlastného riešenia crawlera s modulárnou architektúrou, nízkymi nárokmi na dlhodobú prevádzku a jednoduchým pridávaním zdrojových stránok. Je schopný zotaviť sa po páde a pokračovať kde prestal. Toto riešenie je možné použiť aj pre distribuované crawlovanie, avšak každý výpočtový uzol musí skúmať unikátnu množinu domén. 

% Tento návrh sme implementovali a úspešne pripravili dáta do vhodnej podoby na analýzu v nástroji Apache Spark. 




% Analyzovaním dát nachádzajúcich sa na spravodajských weboch vieme sledovať nálady v spoločnosti, identifikovať stránky šíriace lživé správy alebo skúmať vývoj slovenského jazyka. 

% Na to ale najprv potrebujeme tieto dáta zozbierať a pripraviť do vhodnej podoby. 

% Existuje viacero crawlerov, nastaviteľných na získavanie dát pre tieto cieľové účely. Sú ale príliš robustné a vyžadujú si zbytočne veľa zdrojov na prevádzku. 

% Prínosmi tejto práce

% Navrhnuté riešenie je nenáročné na údržbu, vyžaduje málo zdrojov. Je prispôsobené na dlhodobú prevádzku a odolné voči pádom. 

% Distribuované počítanie sa dá dosiahnuť rozdelením domén a spustením na viacerých počítačoch. 

% tejto práce je navrhnutie vlastného crawlera

% a implementovanie web crawlera prispôsobeného na zbieranie informácii so spravodajských webov.  
% The main contribution of the thesis is the design and implementation of a custom web crawler that is tailored to the needs of news websites. The crawler is designed to be highly configurable and customizable, allowing it to adapt to the specific requirements of different news sources. It is also designed to be efficient and scalable, enabling it to handle large volumes of data without impacting the performance of the target websites.

\chapter*{Formulácia úlohy}

% Text záverečnej práce musí obsahovať sekciu s~formuláciou úlohy resp. úloh riešených v~rámci záverečnej práce. V~tejto časti autor rozvedie spôsob, akým budú riešené úlohy a~tézy formulované v~zadaní práce. Taktiež uvedie prehľad podmienok riešenia.

Cieľom práce je návrh a implementácia crawlera, špecializovaného na zber dát zo slovenských spravodajských webov. Hlavnou podmienkou je, aby dáta boli vhodné na spracovanie nástrojom Apache Spark a dala sa nad nimi spraviť rozumná analýza. Riešenie má byť čo najmenej komplexné a má spĺňať iba zadané požiadavky. Teda nemá mať zbytočné funkcionality, čím by sa zhoršila udržiavateľnosť systému. Závislosť na externých systémoch je dovolená iba v nevyhnutnom prípade. Hlavné požiadavky na systém sú: 

\begin{itemize}
    \item Získavanie dát vhodných na analýzu zo spravodajských webov. Konkrétne: titulok, úvodný paragraf, hlavná časť článku, mená autorov, deň vydania, deň poslednej modifikácie.
    \item Paralelizmus - riešenie musí prechádzať stránky paralelne, ale zároveň nesmie zahlcovať web servery.
    \item Odolnosť voči pádom - ak sa vyskytne chyba, nesmie crawler stratiť dáta.
    \item Redukovanie cieľovej domény - nastavovania crawlera umožnujú  filtrovanie stránok, určených na prejdenie.
    \item Pridávanie zdrojov - riešenie musí byť navrhnuté tak aby umožnilo jednoduché pridávanie nových domén na extrakciu. 
    \item Nízka komplexita, jednoduché nasadenie a údržba. 
\end{itemize}


Crawler musí byť schopný niekoľko-hodinovej až niekoľko-dňovej prevádzky. Teda pri vyhodnotení splnenia cieľov práce musíme preskúmať, či crawler nestráca výkon. Crawler teda musí priebežne zbierať štatistiky o trvaní hlavných operácii. Na ich základe určíme trend a závažnosť prípadnej degradácie. 

Súčasťou práce bude aj demonštratívne použitie zozbieraných dát pre jednoduchú analýzu v nástroji Apache Spark. Tým overíme, že riešenie zbiera dáta tak, ako sme to sformulovali v úlohe.
% !TEX root = ../thesis.tex

\chapter{Analytická časť}

Analytická časť záverečnej práce analyzuje existujúce podobné prístupy k~riešeniu stanoveného problému. Autor práce musí uviesť v~tejto časti existujúce prístupy a riešenia, pričom musí zaujať stanovisko k~týmto prístupom a riešeniam a opísať ich výhody a nedostatky. Prevažne v~tejto časti autor používa odkazy na použité zdroje. Autor v~analýze nepreberá odseky z~cudzích prác ale uvádza prevažne vlastné postoje podložené odkazmi na literatúru. Analytická časť práce by teda nemala byť len povrchným prepisom základných informácií z~Wikipédie alebo zo stránok opisovaných nástrojov. Je potrebné aby bola analýza podporená aj experimentmi ak to umožňuje téma práce (napr. vyskúšam softvér). Vďaka popisu existujúcich riešení autor pochopí problematiku, viac sa nad riešeniami zamyslí, usporiada si ich, zistí ich kladné a záporné vlastnosti, z~čoho potom postupne vyplynie návrh vlastného riešenia v~syntetickej časti. Analytická časť tvorí zvyčajne ¼ jadra práce.

Analytickú časť je možné rozdeliť na niekoľko kapitol, ktoré budú venované rôznym analyzovaným témam. Názvy kapitol majú zodpovedať tomu, čo je v~kapitole opisované. Napríklad ak v~práci analyzujete súčasný stav v~oblasti medzigalaktických letov, namiesto všeobecného názvu "`Analýza súčasného stavu"' by mal byť použiťý názov analyzovanej témy --- "`Medzigalaktické lety"'.

% lorem ipsum
\section{Lorem ipsum}
\blindtext

\section{Aliquam eu malesuada urna}
\blindtext
\begin{itemize}
    \item v~knihe \cite{book} autor prezentuje naozaj odvážne myšlienky
    \item nemenej zaujímavé výsledky publikuje ďalší autor v~článku \cite{article} 
    \item v~konferenčnom príspevku \cite{conference} sú uvedené tiež zaujímavé veci
    \item \LaTeX{}\footnote{\url{https://www.latex-project.org/}} je typografický jazyk
\end{itemize}

Given a set of numbers, there are elementary methods to compute its \acrlong{gcd}, which is abbreviated \acrshort{gcd}. This process is similar to that used for the \acrfull{lcm}.

\subsection{Donec vehicula consequat}
\blindtext

\begin{figure}[!ht]
    \centering
    \includegraphics[width=.9\textwidth]{figures/tugboat}
    \caption{\LaTeX{} Friendly Zone \label{o:latex_friendly_zone}}
\end{figure}

\subsection{Nullam in mauris consectetur}
\blindtext

\begin{lstlisting}[language=C,caption={Program, ktorý pozdraví celý svet}]
#include <stdio.h>
int main() {
    /* Print Hello, World! */
    printf("Hello, World!\n");
    return 0;
}
\end{lstlisting}


\subsection{Vestibulum tristique elementum varius}
\blindtext

\begin{table}[!ht]
	\caption{Country list}\label{t:1}
	\smallskip
	\centering

	\begin{tabular}{ |p{3cm}||p{3cm}|p{3cm}|p{3cm}|  }
		\hline
		\multicolumn{4}{|c|}{Country List} \\
		\hline
		Country Name or Area Name& ISO ALPHA 2 Code &ISO ALPHA 3 Code&ISO numeric Code\\
		\hline
		Afghanistan & AF & AFG & 004\\
		Aland Islands & AX & ALA & 248\\
		Albania & AL & ALB & 008\\
		Algeria & DZ & DZA & 012\\
		American Samoa & AS & ASM & 016\\
		Andorra & AD & AND & 020\\
		Angola & AO & AGO & 024\\
		\hline
	\end{tabular}
\end{table}


\section{Phasellus id pretium neque}
\blindtext

\blindtext

\section{Čo rozumieme pod pojmom Web Crawler}

Web Crawler, tiež nazývaný ako Spider, je webový bot, ktorý na základe jedného alebo viacerých východiskových URL adries stiahne webové stránky s nimi spojené, extrahuje hyperlinky obsiahnuté na týchto stránkach a následne pokračuje rekurzívne v sťahovaní webových stránok identifikovanými týmito hyperlinkami. Okrem toho tiež zaznamenáva štruktúru prepojení medzi stránkami. \cite{introToInfRetrieval}

Ich implementácia predstavuje významné inžinierske výzvy v dôsledku rozsahu internetu. Na to, aby prešli podstatnú časť "povrchového webu" v primeranej dobe, musia Web Crawleri sťahovať tisíce stránok za sekundu a zvyčajne sú distribuované na desiatky alebo stovky počítačov. Ich dve hlavné dátové štruktúry - "frontier" sada URL adries, ktoré ešte neboli prehliadané, a sada objavených URL adries - zvyčajne nevojdú do hlavnej pamäte, takže je potrebné používať efektívne reprezentácie na disku. \cite{encykOfDatabases}


\section{Použitie Web Crawlera}
Web Crawleri sú dôležitou súčasťou internetových vyhľadávačov, kde slúžia na zozbieranie korpusu webových stránok indexovaných vyhľadávačom. Pri čom nie je dôležitý iba obsah stránok ale aj vzťahy medzi nimi. Okrem toho sa používajú v mnohých ďalších aplikáciách, ktoré spracovávajú veľké množstvá webových stránok, ako sú ťažby dát z webov, porovnávacie nákupné enginy \footnote{ napríklad \url{https://www.heureka.sk/}} (comparison shopping engines). \cite{encykOfDatabases}.  

Taktiež na analýzu konkurencie pre dynamické určovanie ceny produktov. Napríklad úprava ceny na základe obsadenosti vstupných lístkov u resalarov (aero-linky). 
Môžu byť použité tiež na monitorovanie zmien a automatickú údržbu web-stránok, zber dát pre výskum a sledovanie trendov v spravodajstve. 

\section{Základný algoritmus crawlovania}
Najjednoduchší algoritmus je sekvenčný, slúži ako základ pre zložitejšie algoritmy využívajúce paralelizmus. 

Algoritmus začína pridaním štartovacích URL adries do fronty. Po jednej sťahuje webové stránky priradené k týmto adresám. Extrahuje z nich URL adresy. Tie pridáva do fronty a stiahnuté stránky ukladá do repozitára.

Tento proces končí vyprázdnením fronty. To v praxi nemusí nastať pretože typické použitia web crawlera si vyžadujú neustále obnovovanie extrahovaných informácii.

\begin{figure}[!ht]
    \centering
    \includegraphics[width=.9\textwidth]{figures/basicCrawlAlgorithm.png}
    \caption{Diagram naivného algoritmu \label{o:basic_crawl_algorithm} \cite{dataMining}}
\end{figure}

\section{Fronta} \todo{Premenovat nadpis}
Fronta je centrálny prvok crawlera. Jej implementácia a nastavovanie odráža hlavné vlastnosti systému. 

Naivná implementácia požaduje FIFO dátovú štuktúru s deduplikovaním. Napríklad uchovanie navštívených adries. Pre malé, napríklad POC riešenia je to dostačujúce. 
Všeobecný crawler určený na široký web je typicky distribuovaný a potrebuje stránky znovu navštevovať pre zaručenie aktuálnosti zozbieraných informácii. Taktiež nemá dostatočné zdroje na pokrytie celej domény a musí selektovať adresy na navštívenie. Preto nepostačuje naivné riešenie, a fronta pre praktické využitie musí riešiť tieto požiadavky. 

\section{Politiky prehľadávania}
\textbf{Politika výberu} (selection policy) - určuje výber stránok podľa ich relevantnosti, popularite, aktuálnosti a kvalite. Taktiež pomocou nej vieme zúžiť prehľadávanú doménu, ak našim cieľom nie je celý web (napríklad iba e-shopy). Nikto nemá zdroje na prejdenie kaŽdej verejne dostupnej stránky, musí sa teda vybrať podmnožina s čo najrelevantnejšími informáciami. Preto je politika výberu kľúčová pre efektivitu crawlera. 

\textbf{Politika znovu navštívenia} (re-visit policy) - určuje ako často môže byť stránka znovu navštívená s cieľom získania aktuálnych informácii. Vo všeobecnosti platí, že čím vyššia frekvencia tým je menšia prehľadávaná podmnožina. 

\textbf{Politika zdvorilosti} (Politeness policy) - stránka si môže určiť, maximum a frekvenciu simultánnych requestov, taktiež sekciu stránok neprístupné pre botov. A to v súbore robots.txt \cite{robotsTxt}. 

\textbf{Politika paralelizácie} (Parallelization policy) - určuje, ako crawler rozdeľuje prácu a koordinuje súbežné prehľadávanie stránok. 

Politiky môžu byť statické - nastavenie pre jednotlivé stránky alebo skupiny stránok na základe ich domény. Napríklad zoznam stránok, ktorým dôverujeme a sú pre nás najpodstatnejšie. \todo{dobry nazov - napriklad index.sme.sk sme je domena})
Alebo dynamické - meniace sa podľa informácii získaných prehľadávaním. Napríklad analýzou obsahu stránky určíme relevantnosť stránok v tej istej doméne. 




\section{Vhodné vlastnosti všeobecného crawlera}

\textbf{Odolnosť} - voči pasciam proti botom alebo zle nadizajnovaným stránkam. Napríklad stránky generujúce nekonečnú množinu pod stránok, môžu zahltiť naivný crawler. 

\textbf{Škálovateľnosť} (scalability) - webový priestor sa rýchlo zväčšuje. Aby sa udržal podiel preskúmaných stránok, systém sa musí 
vedieť prispôsobiť. 

\textbf{Distribuovanosť} -  moderný web je mohutný a rýchlo sa meniaci. Pre získanie aktuálneho obrazu, potrebujeme obrovské množstvo dotazov za sekundu, čo vyžaduje distribuovaný systém. 

\textbf{Výkon a efektívnosť} - beh crawlera aj na obmedzenej doméne, vyžaduje robustný systém. Bez dôrazu na efektívnosť sa nám s obmedzenými prostriedkami nepodarí udržať požadovaný pomer spracovaných stránok z celej domény.

\textbf{Prioritizácia} - aplikovanie politiky výberu.

\textbf{Aktuálnosť} - aplikovanie znovu navštívenia.

\textbf{Rozšíriteľnosť} - systém potrebuje spracovávať nové dátové formáty a protokoly. Architektúra by preto mala byť modulárna. 

\textbf{Etickosť} - systém by mal rešpektovať pravidlá v robots.txt.


\section{Existujúce crawlery pre naše použitie} \todo{asi premenovat}

% !TEX root = ../thesis.tex

\chapter{Návrh riešenia}
\label{methodology}

\section{Kontext využitia}

Nami vytvorený crawler bude zbierať dáta z vybraných slovenských spravodajských webov. Relevantné dáta uloží vo vhodnej forme pre následnú analýzu. Interagovať s ním bude iba jeden správca, schopný upravovať jeho kód. Riešenie teda nevyžaduje flexibilitu, zameriavame sa na jedno konkrétne použitie. 

\subsection{Analyzovanie zozbieraných dát}
Analýza dát nie je súčasťou a zameraním tejto práce. Jej popísanie, pre účely návrhu crawlera, považujeme za dôležité.

Dáta na analýzu budú konzumované ETL pipelinou postavenou nad Apache Sparkom prípadne nad platformou Databricks (cloudová nadstavba Sparku). Očistené dáta budu analyzované Apache nástrojmi SparkML (Machine learning) a OpenNLP (natural language processing).

Tieto frameworky podporujú jazyky Java, Scala, Python a R. Scalu považujeme za jasného favorita na budovanie robustných ETL procesov. A to hlavne vďaka jej plnej podpore a dobrej integrácii funkcionálnej paradigmy. Preto bude použitá v tejto časti.

Pre tento ETL proces uloženie dát v relačnej ani inej databáze neprináša žiadne benefity oproti uloženiu v jednom alebo viacerých klasických súboroch ako CSV, Parquet a podobne. 

Predpokladané zameranie analýzy bude sledovanie trendov, sentiment spoločnosti, klasifikácia do tém alebo identifikovanie najrelevantnejších článkov, napríklad pomocou backlink analýzy. 

V čase navrhovania crawlera, nám nie je detailne známe aké dáta si bude vyžadovať analýza. Považujeme to za miesto potencionálneho rozširovania funkcionality. 

\subsection{Nasadenie a zdroje}
Crawler by mal byť spúšťaný pravidelne, predpokladáme raz za mesiac. Prejsť by mal zopár slovenských spravodajských webov a zozbierať dáta na nasledujúce analyzovanie. 

Zdroje na infraštruktúru a údržbu sú veľmi malé. Predpokladáme, jedného správcu, s pár hodinami času do mesiaca. To musíme zohľadniť pri voľbe komplexnosti riešenia. 

Výpočtové a finančné zdroje sú taktiež minimalistické. Predpokladáme, že minimálne zber dát bude nasadený vo virtuálnom operačnom systéme bežiacom na fakultnom serveri. 



\section{Požiadavky}

\subsection{Paralelizmus}
Čakanie na odpoveď servera je hlavné výkonnostné obmedzenie. Preto požadujeme aby systém spracovával stránky paralelne. Týmto výrazne zvýšime efektívnosť a výkonnosť crawlera. 

\subsection{Odolnosť voči pádom}
Predpokladáme, že systém bude bežať pár desiatok hodín. Nevieme zaručiť spoľahlivosť prostredia, v ktorom bude nasadený. Preto musíme rátať s možnými pádmi celého systému. 

Nechceme mrhať zdrojmi a časom, preto vyžadujeme aby systém bol schopný pokračovať v mieste kde skončil. Minimálne pokračovanie od posledného kontrolného bodu (checkpoint).

V ideálnom prípade by tento zotavovací mechanizmus mal byť nezávislí od prostredia. A zotaviť systém aj po preinštalovaní virtuálneho OS. Napríklad využitie služby DaaS (database as a service). Chápeme ale požiadavku na nízke náklady, ľahké nasadenie a jednoduchú údržbu. Preto sa uspokojíme aj s riešením v rámci jedného OS. Ale chceme aby riešenie bolo možné ľahko modifikovať na externý systém ukladania kontrolných bodov.

Za vhodné považujeme aj logovanie s nastaviteľnou úrovňou, pre zjednodušenie hľadania možných chýb.

\subsection{Obmedzená doména}
Zameriavame sa na extrahovanie dát z vybranej skupiny spravodajských webov. Túto skupinu chceme jednoducho upravovať, či už pridávať nové zdroje alebo redukovať existujúce. Táto časť riešenia by mala byť otvorená rozširovaniu. 

\subsection{Nízka komplexita, jednoduché nasadenie a údržba}
Potrebujeme aby systém bol ľahko udržateľný a nasaditeľný. Preto požadujeme aby systém bol čo najmenej komplexný. Znížená robustnosť a menej dostupných funkcionalít nám neprekáža. Potrebujeme najjednoduchšie riešenie čo zvládne vyriešiť náš problém, s čo najmenej zdrojmi (lightweight software). 

Očakávané úlohy údržby: \todo{formátovanie aby pekne sedelo}
\begin{itemize}
  \item Pridávanie a odoberanie cieľových domén.
  \item Oprava chýb.
  \item Úprava formátu a cieľa zozbieraných dát.
  \item Výber zbieraných dát.
  \item Spúšťanie nasadeného riešenia. 
\end{itemize}


\subsection{Požadované dáta na extrakciu}
\begin{itemize}
  \item Názov článku
  \item Úvodný paragraf, zhrňujúci článok.
  \item Hlavná časť článku.
  \item Mená autorov.
  \item Deň vydania. 
  \item Deň poslednej modifikácie.
\end{itemize}

Ako sme spomínali, očakávame úpravu požadovaných dát. Ako aj zbieranie doménovo unikátnych dát, teda dáta, ktoré nebudú musieť byť extrahované z celého korpusu prehľadávaných článkov (napr. komentáre článku). Neprekáža nám jednotný dátový formát, s neplatnými hodnotami v miestach nepodarenej extrakcie. 
Považujeme to za jedno z hlavných miest možného rozširovania systému, teda tieto zmeny musia byť robené rýchlo a jednoducho. 



\section{Ukladanie dat}
Pre naše účely sú ACID vlastnosti zbytočné. Momentálne nevieme o výhode Preto navrhujeme aby crawler extrahované dáta ukladal do týchto súborov. 


\chapter{Implementácia}
\label{implementation}

V tejto kapitole podrobne popíšeme implementáciu podla návrhu v predchádzajúcej kapitole. \todo{Dopisat uvod ked bude kapitola hotova}

\begin{figure}[!ht]
    \centering
    \includegraphics[width=.9\textwidth]{figures/classDiagramVisitResult.png}
    \caption{Class diagram pomocných dátových štruktúr \label{o:classDiagramVisitResult}}
\end{figure}

\section{Implementacia modulov}
V návrhu architektúry sme rozdelili riešenie do 4 základných modulov a popísali ich zodpovednosť a základnú komunikáciu medzi nimi. Taktiež sme popísali prístup k paralelizácii. V tejto podkapitole na to nadviažeme a popíšeme ako sme implementovali tieto moduly. 

\subsection{Orchestrátor}
Tento modul je vlastne jadro aplikácie, integrujúci zvyšné moduly. Tie komunikujú výlučne s ním a nie medzi sebou. Táto separácia nám umožnila jednoduché testovanie jednotlivých modulov a sledovanie komunikácie medzi nimi.

V implementácii sme nepoužili priamo pojem orchestrátor, ale implementovali sme ho v triede Crawler. Tá reprezentuje celú našu aplikáciu a jej metóda \textit{crawlMainJob} \todo{Ako pomenuvavat metody, oznacit ze ide o nazvy z kodu} štartuje samotný crawling. 

Pre jej inicializáciu musíme dodať objekt triedy CrawlerContext, obaľujúci inštancie zvyšných modulov. Takto vieme spúšťať crawler s rôznymi implementáciami modulov a tým meniť jeho správanie. Ide vlastne o známy princíp injektovania závislostí (dependency injection). \todo{Ak treba popisat benefity}

Po spustení sa vykonáva krok crowlovania, implementovaný funkciou doStep. Čo sa vykonáva v kroku je popísané v návrhu. Za zmienku ale stojí, že rozdelenie práce pracovným vláknam a agregovanie výsledku je presunuté do funkcie crawlStep. Tá pre pole adries vráti objekt CrawlResult. Bližšie si jej implementáciu popíšeme v samostatnej podkapitole. 

Návratovou hodnotou doStep je trieda RunStats. Tú priebežne agregujeme do CrawlerRunReport. Ide napríklad o počet navštívených adries a počet nezdarov. Využili sme triedy ako BigInt, kedže rátame s dlhodobým behom. 

\subsection{Funkcia crawlStep}
Táto funkcia rozdelí poskytnuté adresy parametrom do skupín. Každá skupina je potom spracovaná jedným z pracovných vlákien. Počet skupín je v štandardnom nastavení väčší ako počet pracovných vlákien. Tým zvýšime vyrovnanosť záťaže jednotlivých vlákien.

Využili sme abstrakciu nad paralelným spracovaním - \textit{Future}. Ide o generickú triedu štandardnej scala knižnice reprezentujúca budúcu hodnotu. V našom prípade budúcu hodnotu CrawlResult. Z pohľadu funkcionálneho programovanie je to monád. Tento prístup sa stáva populárnym aj v nie plno funkcionálnych programovacích jazykoch ako Java, Python a podobne. 

Práca vlákna je reprezentovaná funkciou crawlChunk. Tá mapuje skupinu adries na budúcu hodnotu CrawlResult. 

Funkcia crawlStep čaká na všetky budúce hodnoty, následne agreguje čiastkové výsledky práce do jedného. Reprezentujúceho výsledok navštívenia a extrahovania stránok určených na spracovanie v tomto kroku. 

Teda všetká paralelná činnosť programu je izolovaná v tejto funkcii. 

\subsection{Repozitár}
Modul repozitár reprezentuje interface s jednou metódou - saveStep. Jej jediný parameter je pole objektov RepoDTO (Data Transfer Objekt). Tento objekt reprezentuje schému ukladaných dát a je nezávislý na reprezentácií týchto dát vo zvyšku programu - PageResult. 

Táto striktná nezávislosť nám umožní v budúcnosti zmeniť spôsob ukladania výsledkov napríklad do relačnej databázy bez zásahu do zvyšku programu. Vytvoríme iba novú implementáciu tohto interfacu a vložíme ju do objektu CrawlerContext. 

Vytvorili sme 2 základné implementácie a jednú určenú na testovanie. Prvá ukladá výsledky všetkých krokov do jedného CSV súboru. To je vhodné pre krátke behy hlavne pri optimalizácii parametrov. Druhá ukladá každý krok do samostatného CSV. Pre produkčné nasadenie používame druhú implementáciu. 

Implementácia určená na testovanie ukladá objekty RepoDTO iba do vnútorného poľa. A pridáva metódu getAllSaved. To využívame v integračných testoch. Takto vieme skontrolovať či sa ukladajú do repozitára požadované výsledky a nemusíme pozerať do súborového systému. 


\subsection{Url Manažér}
Tento modul je reprezentovaný interfacom s metódami:
\begin{itemize}
    \item Upsert pridá Url adresy na prejdenie v budúcich krokoch. 
    \item GetBatch vráti adresy na prejdenie v aktuálnom kroku. 
    \item markAsCrawled označí adresy ako prejdené
\end{itemize}


Pri návrhu sme sa obávali, že nám uchovanie v operačnej pamäti nebude stačiť a budeme nútený použiť databázu. Ale testovanie aplikácie nepotvrdili túto obavu. Teda pre zníženie náročnosti na údržbu a infraštruktúru sme sa rozhodli pre jednoduchšie riešenie.

Hlavná implementácia tohto modulu ukladá potrebné dáta v operačnej pamäti a po každom volaní metód upsert a markAsCrawled je serializovaná na disk. Ako sme spomínali v návrhovej časti práce, zodpovednosťou tohto modulu je uchovanie stavu aj napriek pádom. 


Riešenie je pripravené na rozšírenie. V prípade potreby stačí implementovať tento modul robustnejšie a vložiť ho do CrawlerContext. 

\subsection{Extraktor}
Extraktor je implementovaný funkciou crawlUrl. Tá využíva interface DomainScraper s metódami na extrahovanie dát z html  dokumentu (parseDocument) a rozhodnutia či adresa je v danej doméne. Výsledkom extrahovania dát zo stránky je objekt PageContent. 

Následne funkcia crawlUrl odfiltruje podporované dcérske adresy pomocou triedy DomainFilter. Tie spolu s ďalšími hodnotami vráti v objekte CrawlResult. 

Crawled obsahuje teda iba pole podporovaných adries. PageContent na druhú stranu obsahuje všetky adresy nájdené na stránke. 

Pre pridanie podpory extrahovania z novej domény je potrebné implementovať DomainScraper a túto implementáciu pridať do zoznamu podporovaných domén. 


\section{Využitie parciálnej aplikácie funkcií vyššieho rádu}
Parciálna aplikácia je technika funkcionálneho programovania na zvýšenie znovu použiteľnosti funkcií, čitateľnosti a v našom prípade hlavne injektovania závislostí. Funkcia vyššieho rádu je funkcia berúca inú funkciu ako parameter alebo ako návratovú hodnotu. 

My sme tieto techniky využili hlavne pre injektovanie extraktora. Tento modul sme implementovali ako funkciu crawlUrl. Tá mapuje url na CrawlResult. Potrebuje získať html zo servera a extrahovať z neho relevantné dáta. 

Pre neprodukčné účely ako napríklad testovanie je pripájanie na server nežiadúce a chceme deterministickú implementáciu tejto funkcie. Tú vieme podsunúť ako parameter funkcii vyššieho rádu crawlChunk a crawlStep. Bližšie to popíšeme v podkapitole o testovní crawlera. 

\section{Zastavenie crawlera}
Za zastavenie crawlera mimo vyčerpania prehľadávanej fronty je zodpovedný interface CrawlLimit s metódou: shouldStopCrawling. Imeplementáciu poskytneme crawleru pri jeho inicializácii, takto vieme meniť podmienky zastavenia bez zásahu do kódu crawlera. Vytvorili sme 3 implementácie: 

\begin{itemize}
    \item StepLimitedCrawl - limitujeme počtom krokov.
    \item TimeLimitedCrawl - limitujeme ubehnutým časom od inicializácie, zastavíme crawlovanie až po skončení posledného kroku. 
    \item InfiniteCrawl - crawlovanie bez limitu.
\end{itemize}


\section{Testovanie}
Vďaka vhodnej architektúre môžeme každý modul mimo orchestrátora otestovať zvlášť. A orchestrátor otestovať spomínaným injektovaním závislostí. Takto poskytneme testovacie implementácie modulov pre izolované testovanie správania  orchestrátora. Využívame jednotkové testy (unit tests). 

Integračnými testami testujeme hlavný scenár krolovania, teda či všetky moduly spolu komunikujú korektne. 

Beh crawlera je riadený odpoveďami zo serverov. Pre testovanie teda musíme nahradiť tieto odpovede. Posunuli sme to kúsok ďalej a nahrádzame celý extractor. Teda priradenie CrawlResult zadanej adrese. Práve na to sme využili spomínané parciálne aplikovanie funkcii vyššieho rádu.

Takto vieme vytvárať testovacie scenáre jednoducho, rýchlo a bez potreby písania html súborov. Stačí nám pre každý test vytvoriť funkciu priraďujúcu CrawlResult zadanej adrese. Príklad takejto funkcie uvádzame nižšie. Týmto spôsobom otestujeme takmer celú aplikáciu.

\begin{lstlisting}
    (u: Url) => u match {
        case "aaaa" => CrawlResult().addCrawled(crawled1)
        case "bbbb" => CrawlResult().addCrawled(crawled2)
        case "xxx" => CrawlResult().addFailed(Failed(u, "Url not supported"))
      }
\end{lstlisting}


Extraktor testujeme zvlášť na zopár pripravených HTML súboroch. Taktiež zopár testov sa priamo pripája na servery. Je ich malý počet a vieme ich spúšťať samostatne od hlavných testov. Teda zredukovali sme nezávislosť testov na minimum. 

\section{Logovanie}
Veľkú pozornosť sme kládli na dôkladné logovanie. Rátame s niekoľkohodinovým až pár dňovým behom aplikácie. V prípade pádu programu alebo nájdenia chyby potrebujeme čo najviac informácii o tomto behu.

Využili sme java knižnicu na logovanie ale obalili sme ju našou knižnicou. Pridali sme meranie času od posledného zápisu. Taktiež pohodlné meranie a zápis času behu častí programu. To sme využili pri odhaľovaní miest degradácie výkonu pri dlhých behoch. V príklade nižšie môžeme vidieť použitie pri meraní dĺžky získavania adries na prejdenie v kroku. 

\begin{lstlisting}
    te("Getting Batch"){ctx.urlManager.getBatch(stepMaxSize)}
\end{lstlisting}

Výstup v zázneme vyzerá takto:  "Getting Batch -- Start,
    Getting Batch -- End | Time delta from last timed log (millis): 5 | Now: 2023-05-10T16:20:52.557005500"



\section{Výzvy ktorým sme čelili pri implementácii riešenia}
Prvá implementácia využívala polymorfizmus pre rozdielne správanie tried Crawled a Failed. Napríklad pri zápise do repozitára, alebo pri agregovaní štatistík. 

To ale spravilo tieto triedy zodpovednými za veľkú časť funkcionality crawlera. Od interakcie s repozitárom po ukladanie adries na nasledujúce prechádzanie. Čo je samozrejme zlý dizajn. 

Preto sme sa rozhodli prerobiť aplikáciu a využiť kompozíciu na miesto dedičnosti. Čo nám umožnilo lepšie oddelenie zodpovedností. 
% !TEX root = ../thesis.tex

\chapter{Vyhodnotenie vytvoreného crawlera}
\label{evaluation}

% Vyhodnocovacia časť je kľúčovou časťou záverečnej práce. Tato časť obsahuje vyhodnotenie navrhnutého (vytvoreného) riešenia. Uprednostňované je objektívne vyhodnotenie výsledkov práce, ktoré sa opiera o~meranie a štatistické metódy, prípadne matematické dôkazy. V~prípade nameraných hodnôt musí autor opísať metódu merania, priebeh merania, výsledky a interpretáciu výsledkov v~kontexte riešeného problému a stanovených cieľov. Na základe vyhodnotenia riešenia autor opíše prínosy svojej práce. Vyhodnocovacia časť tvorí zvyčajne ¼ jadra práce.

Pre demonštráciu funkčnosti nášho riešenia sme ho nasadili na 11 hodín a zozbierali štatistiky z jeho behu popísané v kapitole \ref{c:collectingStats} . Ich vyhodnotením rozhodneme, či riešenie je vhodné na plánovanú prevádzku. Taktiež rozhodneme v akej miere sa nám podarilo splniť stanovené požiadavky.

Crawler sme navrhli na dlhodobú prevádzku. Preto sme sa vyhli zväčšujúcim sa dátovým štruktúram. Teda veľkosť využitej pamäte je v čase takmer konštantná. Urobili sme výnimku pre Url manažér modul. Najprv sme plánovali ho implementovať pomocou databázového systému ale pre splnenie požiadavky na nízku komplexitu sme skúsili jednoduchšiu implementáciu a potrebné informácie držíme priamo v operačnej pamäti. Teda s rastúcim prevádzkovým časom rastú aj pamäťové nároky. V kapitole \ref{c:perfDegr} vyhodnotíme, či tento prístup je vhodný pre dlhšie prevádzky a teda či toto rozhodnutie bolo správne. 


\section{Popis nasadenia}
Crawler sme spúšťali dňa 13.5.2023 o 20:00 na 11 hodín. Nastavenia a špecifikácie systému na ktorom bežal popíšeme v nasledujúcich podkapitolách. 

\subsection{Nastavenia crawlera}
Veľkosťou kroku sme nastavili na 1000. A prehľadávané stránky sme redukovali na podstránky týchto domén:

\begin{itemize}
    \item https://index.sme.sk
    \item https://kultura.sme.sk
    \item https://primar.sme.sk
    \item https://korzar.sme.sk
    \item https://kosice.korzar.sme.sk
\end{itemize}

Za štartovacie adresy sme určili:
\begin{itemize}  
    \item https://index.sme.sk/c/23152259/koniec-banictva-na-hornej-nitre-maju-vyriesit-eurofondy-dotacie-vsak-remisova-stale-nespustila.html
    \item https://kultura.sme.sk/c/23164849/attila-mokos-ak-ma-niekto-chce-tak-si-ma-najde-som-staromodny-a-neviem-sa-predat.html
    \item https://korzar.sme.sk/c/23165953/v-parizi-ich-referendum-odmietlo-co-s-nimi-bude-na-vychode-slovenska.html
    
\end{itemize}

\subsection{Špecifikácie systému}

V tabuľke \ref{t:1} je popísaný systém na ktorom sme spustili crawler. Ide o bežný osobný počítač. Parametre súvisiace s pripojením na internet (dowload, upload a ping) sme merali pomocou stránky https://www.speedtest.net/. 
Crawler bol spustený na Java Virtual Machine - Amazon Correto verzie 11.0.19. Skompilovaný bol v Scala verzii 2.13.10. 

\begin{table}[!ht]
	\caption{Špecifikácie systému, na ktorom crawler bežal}\label{t:1}
	\smallskip
	\centering
\begin{tabular}{ | l | c |}
 \hline
 Procesor &  11th Gen Intel(R) Core(TM) i7-1185G7 @ 3.00GHz   1.80 GHz \\ 
 \hline
 Počet jadier & 4  \\  
 \hline
 Typ systému & 64 bitovový \\
 \hline
 Operačná pamäť & 32 GB \\
 \hline
 Operačný systém & Windows 11 \\
 \hline
 Verzia OS & 22621.1635 \\
 \hline
 Typ disku & SSD \\
 \hline
 Download Mbps  & 93.32 \\
 \hline
 Upload Mbps & 94.61 \\
 \hline
 Ping  & 16 \\
 \hline
\end{tabular}
\end{table}

\section{Zbieranie štatistík} \label{c:collectingStats}
Bez možnosti ukladať zbierané štatistiky do operačnej pamäte, rozhodli sme sa ukladať ich do súboru a zároveň časť z nich priebežne agregovať. Oba prístupy si podrobnejšie popíšeme a vyhodnotíme nimi zozbierané informácie v nasledujúcich podkapitolách. 

\subsection{Agregovanie štatistík} \label{c:agrStats}
Na konci každého kroku vypočítame jeho štatistiky. Konkrétne ide o počet navštívených stránok, počet nepodarených pripojení a pre každú zbieranú značku v objekte PageContent počet podarených extrakcií. Ak máme nejaký text v sledovanej značke považujeme ju za úspešne extrahovanú. Tieto údaje priebežne agregujeme. 

Takto vieme vypočítať úspešnosť extrakcie, čo je dôležitá štatistika pri vyhodnocovaní úspešnosti crawlerov. Pomôže nám aj v včasnom identifikovaní potrebnej zmeny logiky extrakcie v prípade zmeneného formátu skúmanej domény. 

Vysoký počet neúspešných pripojení nám indikuje problém s komunikáciou so serverom. Spôsobenú nadmernou paralelizáciou alebo nevhodným extrahovaním adries. 

\subsection{Zaznamenávanie priebežných údajov}
Aby sme mohli vyhodnotiť aj trendy v nazbieraných dátach potrebujeme zaznamenávať aj čiastkové údaje. Využili sme na to uchovávanie záznamov o aktivitách (Logger). Teda údaje sú zapisované do súboru a nezaťažujú operačnú pamäť.

Zaznamenávame rovnaké údaje ako pri agregovaní, popísané v predchádzajúcej kapitole. Navyše pomocou funkcie na časovanie výrazov popísanej v kapitole \ref{c:impl:logging}, zapisujeme a meriame koľko nanosekúnd v každom kroku zaberú tieto aktivity:

\begin{itemize}
    \item Získanie zoznamu adries na prejdenie v danom kroku (UrlManager)
    \item Paralelné získanie výsledkov - crawling (navštívenie stránok a extrakcia dát)
    \item Pridanie nových adries na prehľadávanie (UrlManager)
    \item Uloženie výsledku daného kroku (Repository)
    \item Označenie adries za navštívené (UrlManager)
\end{itemize}

\section{Vyhodnotenie meraní}
Za 11 hodín behu dokázal crawler navštíviť 517 899 a nepodarilo sa mu navštíviť 227 stránok. Čo činí úspešnosť navštívania 99,96\%. Za tento čas zozbieral v 512 krokoch 9 GB dát. 




\subsection{Vyhodnotenie extrahovania dát}
V tabuľke \ref{t:agrMer} sú vyhodnotené agregované štatistiky opísané v kapitole \ref{c:agrStats}. Úspešnosť extrakcie je percentuálny podiel vydarenej extrakcie skúmanej značky oproti celkovému počtu zdarne navštívených stránok. 

Hlavný článok a titulok, najdôležitejšie pozorované značky, sme extrahovali s takmer perfektnou úspešnosťou. Extrahovanie zvyšných značiek je potrebné v ďalších iteráciách vývoja zlepšiť. 




\begin{table}[!ht]
	\caption{Úspešnosť extrakcie skúmaných značiek}\label{t:agrMer}
	\smallskip
	\centering
    \begin{tabular}{|l|c|c|}
    \hline
        & Počet podarenej extrakcie & Úspešnosť extrakcie \\ \hline
        Titulok & 496003 & 95,77\% \\ \hline
        Autori & 186609 & 36,03\% \\ \hline
        Úvodný text & 209424 & 40,44\% \\ \hline
        Hlavný článok & 517899 & 100,00\% \\ \hline
        Dátum vydania & 206030 & 39,78\% \\ \hline
        Dátum poslednej úpravy & 206030 & 39,78\% \\ \hline
    \end{tabular}
\end{table}

\subsection{Degradácia výkonu} \label{c:perfDegr}
Kedže cieľom práce je navrhnutie a implementovanie crawlera schopného dlhodobej prevádzky, musíme vyhodnotiť či výkon systému nedegraduje a keď áno, tak v akej miere. 

V tomto ohľade nepovažujeme repozitár za kritický, pretože v každom kroku ukladáme rovnako veľký objem dát do samostatného súboru a nečítame dáta z predchádzajúcich zápisov.

Naopak Url Manažér si musí držať dáta o už navštívených adresách a v každom kroku perzistovať svoj stav. Teda s pribúdajúcimi krokmi, za zväčšuje objem dát v operačnej pamäti a tak isto objem zapisovaný do súborového systému. Na grafe \ref{o:opTimes} vidíme lineárne zvyšovanie trvania operácii Url Manažéra. 

V tabubuľke \ref{t:degr} je znázornené priemerné trvania operácie označovania prejdených adries na začiatku behu programu a na konci. A ich podiel oproti priemernej dĺžke samotného crawlovanie v celom behu programu, teda časovo najnáročnejšiu operáciu. Pre behy trvajúce hodiny je to dostačujúce. Ak však potrebujeme crawler spustiť na pár dní, treba zvážiť použitie Url Manažéra s databázovým systémom. Tento problém vieme jednoducho vyriešiť rozdelením celej domény na časti a tie spustiť v samostatných inštanciách crawlera. Týmto vieme docieliť aj distribúciu na viac výpočtových jednotiek. 

Na obrázku \ref{o:crawlTimeChart} vidíme, že čas samotného crawlovania sa mimo lokálnych výkyvov nemení. Teda paralelná časť programu funguje ako bola navrhnutá. Tak isto to neindikuje blokovanie žiadostí servermi. 


\begin{table}[!ht]
	\caption{Porovnanie dĺžky trvania operácie UM oproti crolowaniu}\label{t:degr}
	\smallskip
	\centering
    \begin{tabular}{|l|c|c|c|}
    \hline
          & Upsert & Upsert & Crawl \\ \hline
        Kroky & 0-40 & 480 - 520 & 0-520 \\ \hline
        Priemer trvania & 454854212,8 & 4424228686 & 70731074999 \\ \hline
        Podiel & 0,64\% & 6,25\% & 100,00\% \\ \hline
    \end{tabular}
\end{table}


\begin{figure}[!ht]
    \centering
    \includegraphics[width=1\textwidth]{figures/operationsTime.png}
    \caption{Graf závislosti dĺžky trvania operácii od počtu krokov\label{o:opTimes}}
\end{figure}

\begin{figure}[!ht]
    \centering
    \includegraphics[width=1\textwidth]{figures/crawlTimeChart.png}
    \caption{Graf závislosti dĺžky trvania crawlovania od počtu krokov\label{o:crawlTimeChart}}
\end{figure}



\subsection{Počet adries na prejdenie} \label{c:addInteresting}
Očakávali sme prudký nárast adries na prejdenie v začiatku behu programu a následne stabilný lineárny pokles. To sa nepotvrdilo, na obrázku \ref{o:umSizeChart} vidíme, že v približnom strede behu, tento počet znovu začal prudko stúpať. 

Hypotézou bolo že, že sme objavili adresu odkazujúcu na časť webu, ktorá je málo odkazovaná zvyškom webu. Konkrétne v našom prípade sme prechádzali aj doménu https://primar.sme.sk, ale nemali sme ju v štartovacích adresách. Teda až prvým odkazom na túto doménu sme naštartovali jej objavovanie. 

Pre overenie sme pridali stránku z tejto domény ("https://primar.sme.sk/c/23168416/orl-lekar-mandle-su-ako-policajti-chrania-pred-zlom-ktore-by-nas-mohlo-ohrozit.html") do štartovacích adries a znova spustili crawler na 11 hodín. Systém a nastavenia zostali rovnaké. Výsledok sme ale dostali takmer totožný. Potvrdenie alebo vyvrátenie tejto hypotézy si vyžaduje hlbšiu analýzu.

\begin{figure}[!ht]
    \centering
    \includegraphics[width=.9\textwidth]{figures/umSizeChart.png}
    \caption{Graf závislosti počtu adries na prejdenie od počtu krokov\label{o:umSizeChart}}
\end{figure}





\section{Verifikovanie použiteľnosti dát}
Cieľom práce je navrhnutie a vytvorenie crawlera na zozbieranie dát zo spravodajských webov pre následnú analýzu v Apache Spark a ďalších produktoch v tomto ekosystéme. Touto podkapitolou overíme, či zozbierané dáta sú vo vhodnej forme a majú výpovednú hodnotu. 

Analytická časť celého procesu nie je cieľom tejto práce, preto na demonštráciu splnenia cieľa použijeme iba primitívnu analýzu analýza spätných odkazov, implementovanú kódom \ref{code:backlink} v Apache Spark. Pomocou nej sme identifikovali 1000 najviac odkazovaných článkov. Z tohto zoradenia vieme vyčítať najdôveryhodnejšie alebo aktuálne odporúčané články. Ukážka výstupu je na obrázku \ref{o:analRes}.

Dáta sme úspešne analyzovali a nevyskytli sa žiadne problémy. Myslíme, si že sú vhodné aj na zložitejšie analýzy. Preto považujeme tento cieľ práce za splnený.

\begin{lstlisting}[language=Scala,caption={Primitívna analýza spätných odkazov v Apache Spark}]
val df = spark.read
  .option("header", "true")
  .csv(files: _*)

val selectedDf = df
  .withColumn("child_urls", split(col("child_urls"), ", "))
  .filter(col("state") === "Crawled")
  .filter(col("article_text").isNotNull && col("article_text") =!= lit(""))

val explodedDf = selectedDf
  .withColumn("child_urls", explode(col("child_urls")))
  .filter(supportedDoms.foldLeft(lit(false)){case (acc, domStr) =>
    acc || col("child_urls").startsWith(domStr)})
  .dropDuplicates("url", "child_urls")

val groupedDf = explodedDf
  .groupBy("child_urls").count()
  .withColumnRenamed("count", "backlink_count")
  .withColumnRenamed("child_urls", "url")
  .orderBy(desc("backlink_count"))
  .select("backlink_count", "url")

groupedDf.limit(1000).write
  .option("header", "true")
  .format("csv").save(targetPath)
}
\end{lstlisting}\label{code:backlink}

\begin{figure}[!ht]
    \centering
    \includegraphics[width=1\textwidth]{figures/analExample.png}
    \caption{Ukážka výsledku analýzy\label{o:analRes}}
\end{figure}



\section{Splnenie požiadaviek}
Požiadavku na paralelizmus popísanú v podkapitole \ref{c:reqParalel} sme splnili. K stránkam pristupujeme paralelne a výkon sa neznižuje. Nemáme ani problém s blokovaním odpovedí serverom. Teda neprekročili sme maximálny počet dotazov. 

Požiadavku \ref{sec:reqFailRecovery} sme splnili čiastočne. Ukladáme stav Url Manažéra a výsledky crawlovania v každom kroku. Pri páde systému prídeme maximálne iba o jeden krok. Teda systém nie je odolný voči pádu operačného systému so stratou dát v súborovom systéme. Čo as ale podľa popisu požiadavky považuje za dostatočné. 

Pridaním nového extraktora pre novú doménu, implementujeme aj metódu isInDomain. Na jej základe sa filtrujú adresy pridávané do Url Manažéra. Týmto obmedzujeme doménu ako je v požiadavke \ref{c:reqDomain}.

Nie je potrebné nasadzovať databázu alebo iné externé systémy. Systém robí len to na čo bol nadizajnovaný a nie je zaťažený zbytočným balastom. Teda požiadavka \ref{c:reqKomplexity} je splnená. 

Ako je možné vidieť v tabuľke \ref{t:agrMer}, zbieranie dát nedosahuje ideálnu úspešnosť a v ďalších iteráciách je ju potrebné zlepšiť. Nepovažujeme to za chybu dizajnu, ale za nedostatočnú optimalizáciu extraktora. Zbierame požadované dáta, preto požiadavku \ref{c:reqData} považujeme za splnenú. 

% !TEX root = ../thesis.tex

\chapter{Záver}
\label{summary}

% Záver práce obsahuje zhrnutie výsledkov práce s~jasným opisom prínosov a pôvodných (vlastných) výsledkov autora a vyhodnotenie splnenia stanovených cieľov. Je to stručné zhrnutie informácií uvedených v~záverečnej práci. Záver by nemal obsahovať nové informácie.

% V~závere by mal tiež autor poukázať na prípadné otvorené otázky, ktoré sú nad rámec rozsahu práce a mal by odporučiť ďalšie aktivity na pokračovanie pri riešení problému. Rozsah záveru je minimálne 1 celá strana.

V tejto práci sme navrhli a implementovali crawler špecializovaný na zber dát zo slovenských spravodajských webov. Jeho veľkou výhodou oproti použitiu existujúcich riešení je nenáročnosť na údržbu a prevádzku a plná kontrola nad priebehom crawlovania. Je schopný dlhodobej prevádzky a je odolný voči pádom. Ak sa vyskytne chyba, vieme proces opätovne spustiť bez straty stavu crowlovania. 

Architektúru sme zvolili modulárnu čím sme umožnili jednoduchú úpravu správania. Napríklad ak sa v budúcnosti rozhodneme ukladať dáta do iného formátu, implementujeme modul zodpovedný za ukladanie výsledkov a poskytneme jeho inštanciu crawleru pomocou injektovania závislostí.  

Paralelizmus sme izolovali od ostatných modulov. Tie teda nemajú žiadnu paralelizačnú logiku. Vďaka tomu, je údržba nenáročná. 

Program zbiera štatistiky o behu ako napríklad trvanie najdôležitejších operácii a úspešnosť extrakcie dát. Pomocou nich vieme monitorovať nasadené crawlery. 

Vytvoreným riešením sme za 11 hodín zozbierali 9 gigabajtov dát z 518 000 web stránok. Na týchto dátach sme v nástroji Apache Spark vykonali jednoduchú analýzu spätných odkazov\todo{premenvat vsade}. Čím sme identifikovali aktuálne odporúčané články a články s najväčším dosahom. Týmto sme overili splnenie hlavného cieľa práce a to schopnosť zozbierať dáta na podobné účely.  

Zo zozbieraných štatistík sme vyhodnotili, že naše riešenie je vhodné na behy trvajúce hodiny. Avšak po pár dňoch degraduje výkon na nežiadúcu úroveň. Je to z dôvodu ukladania prejdených adries v pamäti a ich perzistovanie na disk v každom kroku. V tomto ohľade navrhujeme pre budúce prácu upraviť perzistovanie z kompletnej serializácie v každom kroku, na perzistovanie iba inkrementov. 

Pri analýze štatistík z behu sme pozorovali prerušenie poklesu adries náhlym prudkým nárastom. V kapitole \ref{c:addInteresting} sa mu povrchovo venujeme. Považujeme ho za zaujímavý úkaz hodný hlbšieho preskúmania v inej práci.  


\todo{ked treba text - testy}

% good linebraking of bibtex url
\setcounter{biburllcpenalty}{7000}
\setcounter{biburlucpenalty}{8000}

%% The bibliography
\printbibliography[heading=bibintoc]

\label{theend} % the last page of the thesis

% List of acronyms
\printglossary[type=\acronymtype,title={\acrlistname}]

% Glossaries
\printglossary

%% Appendix
% !TEX root = ../thesis.tex

\chapter*{\appendixlistname}
\addcontentsline{toc}{chapter}{\appendixlistname}

\begin{description}
	\item[\appendixname{} A] Používateľská príručka
    \item[\appendixname{} A] Systémový manuál
    \item[\appendixname{} C] CD médium -- záverečná práca v~elektronickej podobe,
    % \item[\appendixname{} C] Používateľská príručka
    % \item[\appendixname{} D] Systémová príručka
\end{description}

\appendix
\renewcommand\chaptername{\appendixname}
% !TEX root = ../thesis.tex

\chapter{Príručka pre správcu}

% \section*{Karel's Primitives}

% \begin{itemize}
%     \item \verb|void movek()| - Moves \textit{Karel} one intersection forward.
%     \item \verb|void turn_left()| - Pivots \textit{Karel} $90$ degrees left.
%     \item \verb|void pick_beeper()| - Takes a beeper from the current intersection and puts it in the beeper bag.
%     \item \verb|void put_beeper()| - Takes a beeper from the beeper bag and puts it at the current intersection.
%     \item \verb|void turn_on(char* path)| - Turns \textit{Karel} on.
%     \item \verb|void turn_off()| - Turns \textit{Karel} off.
% \end{itemize}


% \section*{Karel's Sensors}

% \begin{itemize}
%     \item \verb|int front_is_clear()| - Returns \texttt{1} if there is no wall directly in front of \textit{Karel}. \texttt{0} if there is.
%     \item \verb|int right_is_clear()| - Returns \texttt{1} if there is no wall immediately to \textit{Karel}'s right. \texttt{0} if there is.
%     \item \verb|int beepers_present()| - Returns \texttt{1} if \textit{Karel} is standing at an intersection that has a beeper. \texttt{0} otherwise.
%     \item \verb|int facing_north()| - Returns \texttt{1} if \textit{Karel} is facing north. \texttt{0} otherwise.
%     \item \verb|int beepers_in_bag()| - Returns \texttt{1} if there is at least one beeper in \textit{Karel}'s beeper bag. \texttt{0} if the beeper bag is empty.
% \end{itemize}


% \section*{Misc Functions}

% \begin{itemize}
%     \item \verb|void set_step_delay(int)| - Sets delay of one \textit{Karel}'s step in miliseconds.
%     \item \verb|loop(int)| - Repeats \textit{Karel}'s instruction in a loop.
% \end{itemize}


Táto príručka je určená pre správcu crawlera. Rozoberieme si ako crawler inicializovať, monitorovať jeho beh. \todo{doplnenie}

Sme názoru, že dobre napísaný kód je seba dokumentujúc. Preto popíšeme iba princípy a a kroky potrebné pre vykonanie bežných úkonov. 

\section{Príprava prostredia}
Crawler nie je závislí na žiadnom externom systéme. Teda netreba nasadzovať databázu ani nič podobné. 

Je potrebné mať JRE (Java Runtime Environment), ale odporúčame JDK (Java Development Kit) verzie 11. 

\section{Inicializácia}
Pri inicializácii vieme crawleru poskytnúť rôzne implementácie modulov a tým ovplyvniť jeho správanie. Napríklad ak bude potrebné zmeniť ukladanie výsledkov z CSV formátu do iného, stačí implementovať novú verziu repozitára a pri inicializácii ňou naplniť CrawlerContext. Podrobný postup je znázornený na diagrame \ref{o:initChart} a ukážka kódu je na  \ref{o:initCrawl2}

Dôležitou súčasťou nastavení je aj veľkosť kroku (stepMaxSize), udávajúca počet adries spracovaných v jednom kroku. Jej zvýšením vieme zefektívniť beh programu, keďže redukujeme počet neparalelných úsekov a prístup do súborového systému. Na druhú stranu ale zvyšujeme nároky na pamäť, keďže v nej pred uložením držíme práve tento počet výsledkov extrahovania. Teda ak zvo-
líme tento parameter príliš vysoký, program môže vyčerpať pridelenú operačnú pamäť. Oproti tomu menším problémom je aj zväčšenie veľkosti stratených dát pri páde programu.

\begin{figure}[!ht]
    \centering
    \includegraphics[width=1\textwidth]{figures/initChart.png}
    \caption{Diagram inicializácie crawlera\label{o:initChart}}
\end{figure}

\begin{figure}[!ht]
    \centering
    \includegraphics[width=.9\textwidth]{figures/crawlInit.png}
    \caption{Ukážková iniclializácia crawlera\label{o:initCrawl2}}
\end{figure}

\section{Vyhodnotenie štatistík}
Crawler zbiera štatistiky pomocou záznamov ukladaných do súbory. Skript XXXX \ref{nazov} ich potom spracuje a zapíše do CSV súboru. 

Odporúčame každý jeden stĺpec dať do samostatného grafu. Výsznam stĺpcov: 
\begin{itemize}
    \item \textbf{crawlT} - trvanie crawlovacej aktivity - jej plávajúci priemer má byť konštatntný - rastúci trend indikuje problém s pripojením na server zapríčinenú príliš veľkou paralelizáciou (nanosekundy)
    \item \textbf{stepT} - dĺžka jedného kroku - tiež by mala byť konštantná (milisekundy)
    \item \textbf{batchT} - trvanie získania adries na prejdenie v danom kroku (nanosekundy)
    \item \textbf{upsertT} - trvanie pridania nových adries na prejdenie - očakávame mierny lineárny nárast (nanosekundy)
    \item \textbf{upsertT} - trvanie pridania nových adries na prejdenie - očakávame mierny lineárny nárast (nanosekundy)
\end{itemize}


\section{}

\chapter{Systémový manuál}

Tento manuál slúži pre popísanie crawlera na úrovni vhodnej pre úpravu systému a rozširovanie jeho funkcionalít. 

% Scala je vysoko úrovňoví jazyk schopný lepšej reprezentácie abstrakcie ako diagramy a bežná ľudská reč. Táto dokumentácie je určená primárne pre programátora so znalosťou tohto jazyka, preto popisuje len zodpovednosti modulov. 

\section{Moduly}
V tejto podkapitole popíšeme stručne zodpovednosť modulov a ich rozhranie voči zvyšku systému. 

\subsection{UrlManager}
Modul zodpovedný za ukladanie stavu crawlovania a jeho perzistovanie. Hlavná implementácia využíva jednoduché dátové štruktúry - zoznam a množinu na ukladanie stavu na perzistovanie uskutočnuje serializáciou na disk. Ak crawler má bežať nepretržite dni, je potrená nová implementácia. Perzistovanie iba incrementov, nie celého stavu alebo prejdenie k využitiu relačnej databázy. Tej sa chceme ale vyhnúť pre zachovanie jednoduchosti infraštruktúry. Jej rozhranie a implementácie sú znázornené na diagrame tried \ref{o:urlManChart}. 

\begin{figure}[!ht]
    \centering
    \includegraphics[width=1\textwidth]{figures/urlManagerChart.png}
    \caption{Diagram tried UrlManager\label{o:urlManChart}}
\end{figure}


\subsection{Repository}
Modul zodpovedný za ukladanie výsledkov. Vytvorené implementácie sú ukladanie do jedného CSV súboru, teda pridávanie výsledkov na jeho koniec. A ukladanie každého kroku do samostatného CSV súboru. Taktiež sme vytvorili implementáciu, ukladajúcu výsledky do pamäte. Tá je určenú len na testovanie zvyšných modulov. Jej rozhranie a implementácie sú znázornené na diagrame tried \ref{o:repoClassChart}. 

\begin{figure}[!ht]
    \centering
    \includegraphics[width=1\textwidth]{figures/repositoryClassChart.png}
    \caption{Diagram tried Repository\label{o:repoClassChart}}
\end{figure}

\subsection{Modul crawlovania}
Nejde o samostatnú triedu ale o sadu funkcii, jedna volajúca druhú druhú. Je to jediné miesto paralelizácie a je striktne separované od zvyšku systému. Teda zvyšné moduly vôbec nevedia o paralelizácii a nemusia sa tomu prispôsobovať. Jeho úlohou je pre každú pridelenú adresu vrátiť CrawlResult. Na extrahovanie dát z HTML dokumentu a rozhodnutia či nájdená adresa je v prehľadávanej doméne využíva modul Extraktor. \todo{flow diagram volania tried}

\subsection{Extraktor}
V kóde je nazvaný DomainScraper. Slúži extrahovanie obsahu z HTML dokumentu a na určenie či adresa patrí pod jeho doménu. Každá implementácia je zameraná na jednu doménu. Pre pridanie novej domény na extrakciu je potrebné implementovať interface DomainScraper a pridať ho do zoznamu v objekte (scala alternatíva k java singletonu) SupportedDomains. Pre uľahčenie inicializácie crawlera navrhujeme v budúcnosti nastavovanie SupportedDomains cez CrawlerContext. Pre výber správneho extraktora pre spracovávanú adresú slúži DomainFilter s metódou getDomainScraper. Vzťahy medzi týmito triedami sú vyjadrené diagramom \ref{o:domScraperChart}.

\begin{figure}[!ht]
    \centering
    \includegraphics[width=1\textwidth]{figures/domainScraperChart.png}
    \caption{Diagram tried Extraktora\label{o:domScraperChart}}
\end{figure}

\subsection{Koordinátor}
Je to hlavná časť programu, sprostredkúvajúca komunikáciu medzi modulmi. V kóde je implementovaný ako trieda MyCrawler. 




\section{Pomocné triedy}
V programe používame triedy, ktoré nemôžeme považovať za samostatné moduly. V tejto podkapitole popíšeme stručne ich zodpovednosťa ich rozhranie voči zvyšku systému. 

\subsection{CrawlerContext}
Je to trieda slúžiaca na injektovanie závislostí crawleru. Teda aby sme mohli pri inicializácii crawlera určiť aké implementácie modulov chceme použiť a tým jednoducho meniť jeho správanie. Napríklad pri integračných testoch používať testovacie implementácie. Jeho štruktúru môžeme vidieť na diagrame tried \ref{o:classDiagramContextManual}.

\begin{figure}[!ht]
    \centering
    \includegraphics[width=1\textwidth]{figures/classDiagramContext.png}
    \caption{Diagram tried CrawlerContext\label{o:classDiagramContextManual}}
\end{figure}


\subsection{Logger}
Pomocný modul, slúžiaci na robenie záznamov o behu programu. Oproti bežnej implementácii umožňuje meranie trvania operácii. Napríklad ako dlho trvá zápis výsledkov. Takto vieme monitorovať priebeh crawlovania, čo je nápomocné pre jeho optimalizáciu a odhaľovanie chýb. \todo{Mozno class diagram ak bude malo miesta}



\section{Beh programu a paralelizácia}
Beh programu sme rozdelili do krokov. To nám umožnilo úplnú separáciu paralelného výpočtu od zvyšných modulov. Tým sme docielili, že iné moduly nemusia riešiť paralelizačnú logiku. Jeden krok je vyjadrený pomocou sekvenčného diagramu \ref{o:seqStepManual}.

\begin{figure}[!ht]
    \centering
    \includegraphics[width=1\textwidth]{figures/seqDiagCrawlStep.png}
    \caption{Sekvenčný diagram jedného kroku\label{o:seqStepManual}}
\end{figure}




\chapter{Odkazy na výsledky práce}

V tejto prílohe uvedieme odkazy na výsledky práce. 

\subsection{Repozitár s kódom}

https://github.com/MaklaryP/WebScraperBC.git

\subsection{Zozbierané dáta}

https://drive.google.com/file/d/1qATQCr1hUEEQ5DDHCyIybdwzRvfiMfBB

% zivotopis autora
%\curriculumvitae\protect
%Táto časť\/ je nepovinná. Autor tu môže uviesť\/ svoje biografické
%údaje, údaje o~záujmoch, účasti na~projektoch, účasti na~súťažiach,
%získané ocenenia, zahraničné pobyty na~praxi, domácu prax, publikácie
%a~pod.
\end{document}
