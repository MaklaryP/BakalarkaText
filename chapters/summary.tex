% !TEX root = ../thesis.tex

\chapter{Záver}
\label{summary}

% Záver práce obsahuje zhrnutie výsledkov práce s~jasným opisom prínosov a pôvodných (vlastných) výsledkov autora a vyhodnotenie splnenia stanovených cieľov. Je to stručné zhrnutie informácií uvedených v~záverečnej práci. Záver by nemal obsahovať nové informácie.

% V~závere by mal tiež autor poukázať na prípadné otvorené otázky, ktoré sú nad rámec rozsahu práce a mal by odporučiť ďalšie aktivity na pokračovanie pri riešení problému. Rozsah záveru je minimálne 1 celá strana.

V tejto práci sme navrhli a implementovali crawler špecializovaný na zber dát zo slovenských spravodajských webov. Jeho veľkou výhodou oproti použitiu existujúcich riešení je nenáročnosť na údržbu a prevádzku a plná kontrola nad priebehom crawlovania. Je schopný dlhodobej prevádzky a je odolný voči pádom. Ak sa vyskytne chyba, vieme proces opätovne spustiť bez straty stavu crowlovania. 

Architektúru sme zvolili modulárnu čím sme umožnili jednoduchú úpravu správania. Napríklad ak sa v budúcnosti rozhodneme ukladať dáta do iného formátu, implementujeme modul zodpovedný za ukladanie výsledkov a poskytneme jeho inštanciu crawleru pomocou injektovania závislostí.  

Paralelizmus sme izolovali od ostatných modulov. Tie teda nemajú žiadnu paralelizačnú logiku. Vďaka tomu, je údržba nenáročná. 

Program zbiera štatistiky o behu ako napríklad trvanie najdôležitejších operácii a úspešnosť extrakcie dát. Pomocou nich vieme monitorovať nasadené crawlery. 

Vytvoreným riešením sme za 11 hodín zozbierali 9 gigabajtov dát z 518 000 web stránok. Na týchto dátach sme v nástroji Apache Spark vykonali jednoduchú analýzu spätných odkazov\todo{premenvat vsade}. Čím sme identifikovali aktuálne odporúčané články a články s najväčším dosahom. Týmto sme overili splnenie hlavného cieľa práce a to schopnosť zozbierať dáta na podobné účely.  

Zo zozbieraných štatistík sme vyhodnotili, že naše riešenie je vhodné na behy trvajúce hodiny. Avšak po pár dňoch degraduje výkon na nežiadúcu úroveň. Je to z dôvodu ukladania prejdených adries v pamäti a ich perzistovanie na disk v každom kroku. V tomto ohľade navrhujeme pre budúce prácu upraviť perzistovanie z kompletnej serializácie v každom kroku, na perzistovanie iba inkrementov. 

Pri analýze štatistík z behu sme pozorovali prerušenie poklesu adries náhlym prudkým nárastom. V kapitole \ref{c:addInteresting} sa mu povrchovo venujeme. Považujeme ho za zaujímavý úkaz hodný hlbšieho preskúmania v inej práci.  


\todo{ked treba text - testy}